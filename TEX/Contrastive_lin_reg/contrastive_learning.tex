\documentclass[a4paper, 11pt]{article}
\usepackage[czech]{babel}
\usepackage[utf8x]{inputenc}
\usepackage[T1]{fontenc}
\usepackage{float}
%\usepackage[latin2]{inputenc} % pro iso8859-2
%\usepackage[IL2]{fontenc}     % fonty vygenerované pro iso8859-2

%%%%%%%%%%%%%%%%%%%%%%%%%%%%%%%%%%%%%%%%%%%%%%%%%%%%%%%%%%%%%%%%%%%%%%%%%
%% Hlavička ZDE VYPLNIT 
\newcommand{\autor}{Jakub BUREŠ}
\newcommand{\cislomp}{1} % 1,2,3
\newcommand{\zadani}{Homework1}
\newcommand{\rocnik}{2020/2021}
%% Konec hlavičky
%%%%%%%%%%%%%%%%%%%%%%%%%%%%%%%%%%%%%%%%%%%%%%%%%%%%%%%%%%%%%%%%%%%%%%%%%


%\usepackage{enumitem} 
%\setlist{noitemsep, nolistsep}

\usepackage{calc}
\setlength{\textheight}{10in}
\setlength{\textwidth}{6.5in}
\setlength\oddsidemargin{0cm}
\setlength\evensidemargin{(0cm}
\setlength\topmargin{(\paperheight-\textheight-\headheight-\headsep-\footskip)/2 - 1in}

%% matematika
\usepackage{amssymb}
\usepackage{amsmath}

%% grafika
\usepackage{color}
\usepackage{tikz}
\usetikzlibrary{decorations.pathreplacing, patterns}
\usetikzlibrary{arrows.meta,positioning, datavisualization}

\begin{document}
%%%%%%%%%%%%%%%%%%%%%%%%%%%%%%%%%%%%%%%%%%%%%%%%%%%%%%%%%%%%%%%%%%%%%%%%%%%%%%%%
%% Vysázení hlavičky - NEMĚNIT
Zakladni rovnice kterou chceme naprogramovat.
\begin{align}\label{q1q2final}
&\min_{\theta}- \mathbb{E}_{p_{\mathrm{data}}(x,y)}\left[\alpha\log q_{\theta}\left(y|x\right)+ \left(1-\alpha\right)\log q_{\theta}\left(x|y\right) \right]  \\
&\approx	\min_{\theta}- \mathbb{E}_{p_{\mathrm{data}}(x,y)}\left[\alpha\log \frac{\exp\left({f_\theta\left(x\right)[y]}\right)}{\sum_{y'}\exp\left({f_\theta\left(x\right)[y']}\right)}+ \left(1-\alpha\right)\log \frac{\exp\left({f_\theta\left(x\right)[y]}\right)}{\sum_{i=1}^K\exp\left({f_\theta\left(x_i\right)[y]}\right)} \right]
\end{align}
Dle znaceni v clanku, \textbf{Joint Energy models}  
\begin{equation}
	p(x,y) = \frac{\exp{\left(f_{\theta}(x)[y]\right)}}{Z(\theta)}
\end{equation}
kde 
\begin{equation}
f_{\theta}(x)[y] = -E_{\theta}(x,y) 
\end{equation}
a pro normalizacni konstantu plati 
\begin{equation}
	Z(\theta) = \sum_x\sum_y\exp{(-E_{\theta}(x,y))} 
\end{equation}
Jak to premostit na jednorozmernou linearni regresi?
\begin{equation}
p(x,y) =	p(y,x) = p(y\vert x)\cdot p(x)
\end{equation}
\begin{equation}
	p(y\vert x) = \mathcal{N}\left(\theta_0 + \theta_1x, \sigma^2 \right)
\end{equation}
a $p(x)$ muzeme zvolit libovolne? Napriklad pro jednoduchost 
\begin{equation}
	p(x) = \mathcal{N}\left(0, \sigma^2\tau^2\right),
\end{equation}
pricemz uvazujeme $\sigma$ za zname a $\tau$ volime dle potreby. Z cehoz plyne ze 
\begin{equation}
p(x,y) = \mathcal{N}\left(0, \sigma^2\tau^2\right) \cdot \mathcal{N}\left(\theta_0 + \theta_1x, \sigma^2 \right)  =  \frac{1}{2\pi\sigma\tau}\exp\left(-\frac{(y-\theta_0-\theta_1x)^2}{2\sigma^2}-\frac{x^2}{2\sigma^2\tau^2}\right)
\end{equation}
a tudiz
\begin{equation}
	f_{\theta}(x)[y] = -\frac{(y-\theta_0-\theta_1x)^2}{2\sigma^2} - \frac{x^2}{2\sigma^2\tau^2}.
\end{equation}
Nyni dosadme (10) do (2) a pouzijeme pravidla pro pocitani s logaritmy.
\begin{align}\label{q1q2final}
	&\min_{\theta}- \mathbb{E}_{p_{\mathrm{data}}(x,y)}\left[\alpha\log q_{\theta}\left(y|x\right)+ \left(1-\alpha\right)\log q_{\theta}\left(x|y\right) \right]  \\
&\min_{\theta}- \mathbb{E}_{p_{\mathrm{data}}(x,y)}\left[\alpha\log \frac{\exp\left({f_\theta\left(x\right)[y]}\right)}{\sum_{y'}\exp\left({f_\theta\left(x\right)[y']}\right)}+ \left(1-\alpha\right)\log \frac{\exp\left({f_\theta\left(x\right)[y]}\right)}{\sum_{i=1}^K\exp\left({f_\theta\left(x_i\right)[y]}\right)} \right] =\\
& \min_{\theta}- \mathbb{E}_{p_{\mathrm{data}}(x,y)}\left[\alpha\log \frac{\exp\left({-\frac{(y-\theta_0-\theta_1x)^2}{2\sigma^2} - \frac{x^2}{2\sigma^2\tau^2}}\right)}{\sum_{y}\exp\left({-\frac{(y-\theta_0-\theta_1x)^2}{2\sigma^2} - \frac{x^2}{2\sigma^2\tau^2}}\right)}+ \left(1-\alpha\right)\log \frac{\exp\left({-\frac{(y-\theta_0-\theta_1x)^2}{2\sigma^2} - \frac{x^2}{2\sigma^2\tau^2}}\right)}{\sum_{x}\exp\left({-\frac{(y-\theta_0-\theta_1x)^2}{2\sigma^2} - \frac{x^2}{2\sigma^2\tau^2}}\right)} \right] 
\end{align}
potom pro prvni $q$ faktor dostaneme 
\begin{equation}
\alpha\log q_{\theta}\left(y|x\right) = \alpha \left(\frac{(y-\theta_0-\theta_1x)^2}{2\sigma^2} + \frac{x^2}{2\sigma^2\tau^2}\right) -\alpha\log \sum_{y}\exp\left({-\frac{(y-\theta_0-\theta_1x)^2}{2\sigma^2} - \frac{x^2}{2\sigma^2\tau^2}}\right)\\ 
\end{equation}
a pro druhy
\begin{equation}
\left(1-\alpha\right)\log q_{\theta}\left(x|y\right) = (1-\alpha) \left(\frac{(y-\theta_0-\theta_1x)^2}{2\sigma^2} + \frac{x^2}{2\sigma^2\tau^2}\right) - (1-\alpha)\log \sum_{x}\exp\left({-\frac{(y-\theta_0-\theta_1x)^2}{2\sigma^2} - \frac{x^2}{2\sigma^2\tau^2}}\right)
\end{equation}
\end{document}